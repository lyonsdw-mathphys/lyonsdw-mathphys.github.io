% tex image files

\documentclass[leqno]{article}
\pdfoutput=1
\usepackage{graphicx}
\usepackage{color}
\usepackage{amsmath}
\usepackage{amssymb}
\input xy
\xyoption{all}


%this sets up gill (blackboard) letters for reals, complexes, ets.
%  this is appropriate for 10pt (default) document
\font\tengill=msbm10 %scaled 1000
\font\sevengill=msbm7 %scaled 900
\font\fivegill=msbm5 % scaled 700
\newfam\gillfam 
\textfont\gillfam=\tengill
\scriptfont\gillfam=\sevengill
\scriptscriptfont\gillfam=\fivegill
\def\gill{\fam\gillfam}
\def\R{{\gill R}}
\def\C{{\gill C}}
\def\Z{{\gill Z}}
\def\Q{{\gill Q}}

%%%%%%%%%%%%%%%%%%% placement of figures in the margin %%%%%%%%%%%%%%%%%%%%%%%%%
\newcounter{figurecount}
%\renewcommand{\thefigurecount}{\thesection .\arabic{figurecount}}

% command for figures in the margin
% arguments: figure input file name, reference label, figure horizontal
% offset, figure vertical offset, ``Figure'' label horizontal offset,
% ``Figure'' label vertical offset, caption

\newcommand{\marginfig}[7]{
\marginpar{
\refstepcounter{figurecount}      % increment margin figure counter
\label{#2}                        % label the figure
\vspace*{#4}                      % vertical adjustment for figure
\spacer\\
\hspace*{#3}                      % horizontal adjustment for figure
\input{#1}\\
\vspace*{#6}                      % vertical adjustment for ``Figure'' label
\spacer\\
\hspace*{#5}                      % horizontal adjustment for ``Figure'' label
\shortstack{
Figure~\ref{#2}\\ #7
}
\spacer}}
%%%%%%%%%%%%%%%%%%%%%%%%%%%%%%%%%%%%%%%%%%%%%%%%%%%%%%%%%%%%%%%%%%%%%%%%%%%%%%%%

% definitions for algeom text
\newcommand{\Quat}{\mathbb{H}} % <!-- quaternions -->
\newcommand{\Sgroup}{{\rm \bf S}}  % <!-- group of elliptic geometry transformations -->
\DeclareMathOperator{\Rot}{Rot}

%Aut, End, etc. commands
\newcommand{\Aut}{\makebox{Aut}}
\newcommand{\End}{\makebox{End}}
\newcommand{\Hom}{\makebox{Hom}}

% short cut trial
\def\hey{heybaba}

% mapping arrow and isomorphism arrow
\newcommand{\map}{\longrightarrow}
\newcommand{\isomap}{\hspace{1ex}\widetilde{\map}\hspace{1ex}}
\newcommand{\extC}{\hat{\C}}

% ivisible text ``spacer''
\newcommand{\spacer}{\rule[0cm]{0cm}{0cm}}

% load x-y pic
% \input xypic

\newcommand{\ket}[1]{\left| #1 \right\rangle}

\font\biggill=msbm10 scaled  2000
\begin{document}


\begin{center}
{\Large This is for trying out stuff in \LaTeX\ .}
\end{center}


\vspace*{1in}

\medskip

%\input{displacement.pdf_t}
%\input{rise_run.pdf_t}
\scalebox{.8}{\input{rect_areas.pdf_t}}
%\input{stereoproj.pspdftex}
%\input{stereoproj2.pspdftex}
%\spacer\hspace*{-1.5in}\input{rotrefl.pspdftex}

\medskip

%%%% commutative diagram for "welldefinedness" proposition
%% \hfill
%% \xymatrix{
%% X \ar[r]^{f} \ar[d]_{\pi} & Y \\
%% X/\!\!\sim \ar@{-->}[ur]_{\overline{f}}
%% }
%% \hfill\spacer

%%%% directed graph example
%% \hfill
%% \xymatrix{
%%   & b & \\
%% a \ar[ur] \ar[dr]  & & c \ar[ul]\ar[ll]\\  
%%   & d \ar[ur] & 
%% }
%% \hfill\spacer

%%%% commutative diagram for conjugate transformation definition
%% \hfill
%% \xymatrix{
%% X \ar[r]^{f} \ar[d]_{\mu} & X \ar[d]^{\mu}\\
%% Y \ar[r]^{g} & Y
%% }
%% \hfill\spacer

%% %%%% commutative diagram for conjugate transformation definition
%% \hfill
%% \xymatrix{
%%   SU(2) \ar[r]^{C_{iH}} \ar[d] & SU(2)\ar[d]\\
%%   \Sgroup \ar[d] & U(\Quat) \ar[d]\\
%% \Rot(S^2) \ar@{<->}[r] & \Rot(S^2_\Quat)
%% }
%% \hfill\spacer

%% %%%% no maps labeled, commutative diagram for integers mod n distributive law
%% \hfill
%% \xymatrix{
%%   \Z_n\times \Z_n \ar[r] \ar[d] & \Z_n\ar[d]\\
%%   \Z_n\times \Z_n \ar[r] & \Z_n 
%% }
%% \hfill\spacer

%% %%%% commutative diagram for integers mod n distributive law
%% \hfill
%% \xymatrix{
%%   \Z_n\times \Z_n \ar[r]^{\text{  add}} \ar[d]_{(\text{times }[x]) \times
%%     (\text{times }[x])} & \Z_n\ar[d]^{\text{times }[x]}\\
%%   \Z_n\times \Z_n \ar[r]_{\text{  add}} & \Z_n 
%% }
%% \hfill\spacer

\spacer
\vspace*{1in}

%%%% lift of complex inversion via stereographic projection
\hfill
\xymatrix{
  S^2 \ar[r]^{R_{X,\pi}} \ar[d]_{s} & S^2 \ar[d]^{s}\\
  \extC \ar[r]_{\text{ inversion }} & \extC
}
\hfill\spacer

\hfill
\xymatrix{
  (a,b,c) \ar[r]^{R_{X,\pi}} \ar[d]_{s} & (a,-b,-c) \ar[d]^{s}\\
  \frac{a+bi}{1-c} \ar[r]_{\text{ inversion }} & \frac{a-bi}{1+c}
}
\hfill\spacer


\end{document}












